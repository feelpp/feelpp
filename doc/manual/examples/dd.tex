\feelchapter{Domain decomposition methods}
            {Domain decomposition methods}
            {Abdoulaye Samake, Vincent Chabannes, Christophe Prud'homme}
            {cha:dd}

\section{A Really Short Introduction}
\label{sec:really-short-intr}
In mathematics, numerical analysis, and numerical partial differential equations, domain decomposition methods solve
a boundary value problem by splitting it into smaller boundary value problems on subdomains and iterating to coordinate
the solution between the adjacent subdomains. A corse problem with one or fiew unknows per subdomain is used to further
coordinate the solution between the subdomains globally.
\section{A 1D model}
\label{sec:1d-mode}
We consider the following laplacian boundary value problem
\begin{equation}
  \left \{
    \begin{aligned}
      & -u"(x) = f(x) \quad \text{in} \quad  ]0,1[ \\
      & u(0) =\alpha, ~ u(1) = \beta
    \end{aligned}
  \right.
\label{eq:30}
\end{equation}
where $\alpha, \beta \in \mathbb R.$
\subsection{Schwartz algorithms}
\label{sec:schwartz-algorithms}
The schwartz overlapping multiplicative algorithm with dirichlet interface conditions for this problem at $n^{th}$ iteration is given by
\begin{equation}
  \label{eq:31}
  \left \{
    \begin{aligned}
      -u_1"^n(x) & =  f(x) \quad  \text{in}  \quad  ]0,b[  \\
       u_1^n(0) & =  \alpha \\
       u_1^n(b)  & = u_2^{n-1}(b)
    \end{aligned}
  \right.
\qquad \text{and} \qquad
  \left \{
    \begin{aligned}
      -u_2"^n(x) & =  f(x) \quad  \text{in}  \quad  ]a,1[  \\
      u_2^n(1) & =  \beta \\
      u_2^n(a)  & = u_1^n(a)
    \end{aligned}
  \right.
\end{equation}
where $ n \in \mathbb N^*, a, b \in \mathbb R $ and $a < b$. \\
Let $e_i^n = u_i^n-u~(i=1,2)$, the error at $n^{th}$ iteration relative to the exact solution, the convergence rate is given by
\begin{equation}
  \rho = \frac{\vert e_1^n  \vert}{\vert e_1^{n-1}  \vert} = \frac{a}{b}\frac{1-b}{1-a} = \frac{\vert e_2^n  \vert}{\vert e_2^{n-1}  \vert} .
  \label{eq:32}
\end{equation}

\subsection{Variational formulations}
\label{sec:vari-form-1}
find $u$ such that
\begin{equation*}
  \int_0^b u_1'v' = \int_0^b fv \quad \forall v \qquad \text{in the first subdomain} ~\Omega_1 = ]0,b[
\end{equation*}

\begin{equation*}
  \int_a^1 u_2'v' = \int_a^1 fv \quad \forall v \qquad \text{in the second subdomain} ~ \Omega_2 = ]a,1[
\end{equation*}

% \begin{lstlisting}
% /*
% Assembly in the first subdomain $\Omega_1$
% */
%   auto F$_1$ = M_backend->newVector(Xh$_1$);
%   form1( _test=Xh$_1$,_vector=F$_1$, _init=true ) =
%          integrate( elements(mesh$_1$), f*id(v) );
%   F$_1$->close();
%   auto A$_1$ = M_backend->newMatrix( Xh$_1$, Xh$_1$ );
%   form2( _test=Xh$_1$, _trial=Xh$_1$, _matrix=A$_1$, _init=true ) =
%          integrate( elements(mesh$_1$), gradt(u$_1$)*trans(grad(v)) );
%   A$_1$->close();
%   form2( Xh$_1$, Xh$_1$, A$_1$ ) +=
%          on( markedfaces(mesh$_1$, "Dirichlet") ,u$_1$,F$_1$,g)
%          + on( markedfaces(mesh$_1$, "Interface") ,u$_1$,F$_1$,idv(u$_2$));
% /*
% Assembly in the second subdomain $\Omega_2$
% */
%   auto F$_2$ = M_backend->newVector(Xh$_2$);
%   form1( _test=Xh$_2$,_vector=F$_2$, _init=true ) =
%          integrate( elements(mesh$_2$), f*id(v) );
%   F$_2$->close();
%   auto A$_2$ = M_backend->newMatrix( Xh$_2$, Xh$_2$ );
%   form2( _test=Xh$_2$, _trial=Xh$_2$, _matrix=A$_2$, _init=true ) =
%          integrate( elements(mesh$_2$), gradt(u$_2$)*trans(grad(v)) );
%   A$_2$->close();
%   form2( Xh$_2$, Xh$_2$, A$_2$ ) +=
%          on( markedfaces(mesh$_2$, "Dirichlet") ,u$_2$,F$_2$,g)
%          + on( markedfaces(mesh$_2$, "Interface") ,u$_2$,F$_2$,idv(u$_1$));

% \end{lstlisting}

\begin{lstlisting}
template<Expr>
localProblem(element_type& u, Expr expr)
{
  /*
  Assembly of the right hand side $\int_\Omega f(x)v(x)\, \mathrm dx$
  */
  auto F = M_backend->newVector(Xh);
  form1( _test=Xh,_vector=F, _init=true ) =
         integrate( elements(mesh), f*id(v) );
  F->close();
  /*
  Assembly of the left hand side $\int_\Omega u'(x)v'(x)\,\mathrm dx$
  */
  auto A = M_backend->newMatrix( Xh, Xh );
  form2( _test=Xh, _trial=Xh, _matrix=A, _init=true ) =
         integrate( elements(mesh), gradt(u)*trans(grad(v)) );
  A->close();
  /*
  Apply the dirichlet boundary conditions
  */
  form2( Xh, Xh, A ) +=
         on( markedfaces(mesh, "Dirichlet") ,u,F,g);
  /*
  Apply the dirichlet interface conditions
  */
  form2( Xh, Xh, A ) +=
         on( markedfaces(mesh, "Interface") ,u,F,expr);

}
\end{lstlisting}


\section{A 2 domain overlapping Schwartz method in 2D and 3D}
\label{sec:2-doma-overl}

\section{Computing the eigenmodes of the Dirichlet to Neumann operator}
\label{sec:comp-eigenm-dirichl}





%%% Local Variables:
%%% coding: utf-8
%%% mode: latex
%%% TeX-PDF-mode: t
%%% TeX-parse-self: t
%%% x-symbol-8bits: nil
%%% TeX-auto-regexp-list: TeX-auto-full-regexp-list
%%% TeX-master: "../feel-manual"
%%% ispell-local-dictionary: "american"
%%% End:
